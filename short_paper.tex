\documentclass[]{llncs}

\usepackage[utf8]{inputenc}
\usepackage{hyperref}	
\usepackage[parfill]{parskip}
%tweak \url{...}
\usepackage{url}
%\urlstyle{same}

%improve wrapping of URLs - hint by http://tex.stackexchange.com/a/10419/9075
\makeatletter
\g@addto@macro{\UrlBreaks}{\UrlOrds}
\renewcommand\subsubsection{\@startsection{subsubsection}{3}{\z@}%
                       {-18\p@ \@plus -4\p@ \@minus -4\p@}%
                       {0.5em \@plus 0.22em \@minus 0.1em}%
                       {\normalfont\normalsize\bfseries\boldmath}}
\makeatother

\setcounter{secnumdepth}{3}% Number up to \subsubsection
\providecommand{\keywords}[1]{\textbf{Keywords: } #1}

\begin{document}

%#====================================Cover Settings=====================================#%
\title{%
\textit{Identity Management in Healthcare Using Blockchain Technology}}

\author{%
João Santos \and
Pedro Salgueiro,
Vítor Nogueira}

\institute{%
  Universidade de Évora, Portugal
  \\
  \email{m39519@alunos.uevora.pt}
  \\
  \email{pds@di.uevora.pt}
  \\
  \email{vbn@di.uevora.pt}
 }

{\def\addcontentsline#1#2#3{}\maketitle}


%#====================================Abstract===========================================#%

\begin{abstract}
Using the Blockchain technology a system can be created to provide several benefits 
over traditional methods being used in today's Health digital landscape. 
Costs and risks associated with these systems can be reduced and information can become 
transparent and trustworthy to all participants. In this article the technological foundations 
that enable this change are explored and analyzed. 
The system will be presented and its impact in Healthcare discussed.

	\begin{keywords}
  		Blockchain, Health, Identity, Big Data
	\end{keywords}
\end{abstract}


%#===================================Introduction========================================#%
\section{Introduction}

Health is becoming more digital thanks to the widespread availability of computing devices.
More and more medical records are stored on a digital format. 
For storing patient clinical data and their identity in a medical context, 
the Electronic Health Record (EHR) was created.
 
While all this information should benefit both patient and health professionals alike, it is not being handled in an 
effective manner due to problems caused, in part, due to the fragmentation of the patients identity that 
naturally occurs in today's Health Information Systems.

Health is an important topic, for everyone. Healthcare should strive to provide the best service it can for everyone 
and everyone should have access to a quality service. EHR are being generated at an ever increasing rate but most of 
the data is not used in a way that puts the patient's privacy and trust at the forefront.

The purpose of the work presented in this paper is to create and implement a Blockchain based system for Identity Management 
in the Healthcare domain. The patient will be able to manage his data and control its access. Such a system would be suited 
to handle the patient’s identity, for example, in hospitals or clinics and would be able to solve many problems in how data 
is traditionally handled in the Information Systems (IS) available in a regular medical environment.

Blockchain is known as the technology behind the Bitcoin Cryptocurrency, altough nowadays it is being used for many more 
purposes that are explored in the following sections, and its main design goal is to provide security and immutability to 
an agreed upon list of records.

A blockchain runs on a network of computers and the list of records is replicated in some manner depending on the blockchain 
implementation. The first blockchain was conceptualized as the public ledger for the Bitcoin cryptocurrency in 2008 by 
Satoshi Nakamoto, a pen name of, a still unknown to this day, individual or organization of individuals. 
The network was implemented in 2009 and many are now finding it has a much broader potential across many fields, 
with some implementations even resembling a programming platform to execute code in an autonomous manner.
\cite{Nakamoto2008}

A single universal way to identify a person in a given environment is clearly something we should strive towards as seen in, 
for example, \textit{Cartão de Cidadão}, a portuguese identification document that replaces four other identification documents, 
streamlining portuguese civilian identification. 
This also allows many businesses to tailor their services to this document making it easier on both parts and eliminating 
unnecessary costs and risks.

Electronic Health Records (EHR) have seen some progress made regarding the standards that allow for 
interoperability between different organizations thanks to the Health Level 7 (HL7) standard. 
While this standard is growing in use and is represented internationally, Portugal has just started 
the work required to implement it.
\cite{HealthLevel7}

In an effort to make the identity of a patient more secure and transparent a Blockchain can be used to create a 
system that puts at the forefront of its design the patients, breaking conventions in traditional patient data handling.

In this article different Blockchain implementations are explored and related work in this field is presented. 
More precisely, in section \ref{background}, a brief introduction to Blockchain is made followed by an introduction 
to its most prominent implementations. Then a number of real-world use cases of this technology in the healthcare field 
are explored. In section \ref{HLFHealthcare} technical details of a proposed system will be presented. 
Finally, in section \ref{conclusion},  some conclusions are observed regarding the change enabled by these advances.


%#====================================Background========================================#%

\section{Background} \label{background}

\textit{While Blockchain is not a new concept at this point, it is an evolving technology that is 
being used to solve old problems with new approaches. This section will explore the Blockchain technology 
origins and history, some of its different implementations and a brief history to the identity problem is presented.}

\subsection{Blockchain Technology}

A Blockchain can be many things, it can refer to the Bitcoin blockchain, alternative implementations 
or forks of the Bitcoin Blockchain called Altchains or even platforms that allow execution of code 
in an autonomous manner, exactly as it was programmed, with no human intervention. 
Is is a continuously growing list of records, written in the ledger, a structure where records are
written, that is being replicated across a network of devices in opposition to having a 
single central record history, making it a good example of a distributed database.
\cite{Wood2017}
  
The main design goal of the Blockchain is security and to fulfill this purpose it uses techniques 
such as cryptography and digital signatures to not only verify the authenticity of records but also 
read or write access to the network.

Unlike a conventional central data storage, where only a single entity keeps a copy of the 
underlying database, the ledger of the Blockchain is replicated across any number of nodes. 
Not every participant has the same ability to interact with the ledger and in this respect 
a blockchain can be permissionless or permissioned. In a permissionless blockchain every node of 
the network can write in the blockchain whereas in a permissioned blockchain only a select group 
of entities have access to writing in the ledger making the permissioned version, by default, 
secure if the entities themselves are secure and considered trustworthy.

How does a permissionless Blockchain maintain security if every participant has access to 
writing on it, including potentially malicious parties?

Take for example the Bitcoin Blockchain that uses a peer-to-peer network to avoid meddling 
from a financial institution or a third party in a financial transaction. Given that participating 
nodes in the network can belong to different and often competing parties, there is no implied trust 
between them, so the Blockchain needs a mechanism to ensure the integrity of the ledger and prevent 
malicious meddling from interested parties or to avoid a central authority.\cite{Barclay2017}

To solve this problem, consensus mechanisms are used differently, depending on its implementation, 
but having, at its core, a solution to create immutable records and ensure security. 
In Bitcoin Blockchain’s case, consensus is reached by the longest chain rule where the 
longest chain not only serves as proof of the sequence of events witnessed, 
but proof that it came from the largest pool of computing power.\cite{Baars2016}

While the first blockchain was conceptualized as the public ledger for the Bitcoin cryptocurrency 
in 2008 by Satoshi Nakamoto and implemented in 2009, many are now using it as a foundation across 
many application areas such as identity management, traceability and asset management. 
Thanks to the roaring success of Bitcoin and the increasingly apparent use cases that the 
Blockchain can provide, the public awareness of it is rising and it is quickly becoming a 
technological foundation in our economic and social systems.
% Need References for this

\subsubsection{Ethereum}

Bitcoin is getting media coverage almost everyday and public awareness in cryptocurrencies 
in general is rising. 
Some people are considering cryptocurrencies and the Blockchain, to be essentially the same 
technology and, while that may have been somewhat true not so long ago, Blockchain technology 
is starting to be used in a plethora of ways.

Ethereum is an open-source platform based on the Blockchain technology that enables developers 
to build and deploy Decentralized Applications (\textit{DAPPs}).
Ethereum is being developed by the Ethereum Foundation and was first discussed by Buterin [2013]. 
Ethereum intends to provide a blockchain with a built-in programming language that is used to 
create \textit{“Smart contracts”}.
\cite{Wood2017}

These contracts are used to describe the logic of any system that developers can imagine and, 
when created, can then be deployed to the blockchain where they execute as “autonomous agents”. 
Thanks to these tools it is safe to say that long gone are the days where building 
Blockchain applications required a complex background in coding cryptography, mathematics 
as well as significant resources.\cite{Wood2017,BlockGeeks2017}

Ethereum Blockchain is a permissionless Blockchain, and thus, it must have a consensus 
mechanism to ensure the validation process of every record and, in turn, ensure security 
and immutability. While other implementations of the blockchain have different consensus mechanics, 
in Ethereum’s case, all participants have to reach consensus over the order of all transactions 
that have taken place, if a definitive order cannot be established then a double-spend might have occurred.

\subsubsection{Fabric}

Hyperledger Fabric (HLF) is part of the \href{http://www.hyperledger.org/projects/fabric}{Hyperledger} 
project started in December 2015 by the Linux Foundation, and is an open-source developer-focused 
community of communities focused on the development of enterprise-grade, open-source Blockchain-based solutions. 
Fabric is an implementation of a Distributed Ledger Platform (DLP) under the Hyperledger umbrella.
\cite{Cachin2016}

HLF’s initial commit was contributed by IBM and written in Go language. 
It is a permissioned blockchain and its main design goal was to surpass previous 
Blockchain implementation limitations, such as, lack of true private transactions 
and confidential contracts.

This is achieved thanks to assigning peers in the network three distinct roles: 
endorser, committer and consenter and by offering the ability to create channels, 
where a group of participants in the network create a separate ledger. 
HLF is intended as a foundation for developing applications in a modular fashion, 
opting for a plug-and-play approach to various components. \cite{HyperledgerFabricDocs2017}

HLF, as discussed, also allows the creation of smart contracts which can be written in Chaincode. 
As this Blockchain's key operational requirement is privacy, true private transactions 
and confidential contracts can exist and are a great asset for a business environment 
where sensitive information is necessary and disclosed often. 
Thanks to its modular approach consensus protocols are no longer hard-coded and 
trust models can be repurposed.

\subsubsection{Burrow}

Hyperledger Burrow (HLB) is also part of the Hyperledger project and its development 
started in 2014 by Monax and sponsored by Intel. It is a permissionable smart contract machine 
written in Go and offers a modular blockchain client with a permissioned smart contract interpreter 
built, in part, to the specification of the Ethereum Virtual Machine (EVM) and the client has, 
essentially, three main components, the consensus engine, the permissioned EVM and the 
Remote Procedure Call (RPC) gateway.
\cite{Kuhlman2017,HyperledgerBurrow2017}

HLB has its own Consensus Engine, the Byzantine fault-tolerant Tendermint protocol. 
The Tendermint protocol is an open-source effort that allows high performance in 
solving the consensus problem and also has a flexible interface for building 
arbitrary applications above the consensus, as well as, a suite of tools for 
deployments and their management. \cite{Buchman2016}
%
%#===========================Identity in Healthcare===================================#%

\subsection{Identity in Healthcare}
Originally records of a patient were stored in a physical format. 
Thanks to the advent of the computers more and more records are stored on a 
digital format and the Electronic Health Record (EHR) was created. 
This benefits handling of information between the patient 
and the medical professionals and medical institutions.

Standards for EHRs were created and many failed to bring the much needed 
consensus that was required for interoperability between different 
Information Systems in different institutions. 
Health Level 7 has done much work to be recognized in many countries and 
is quickly being implemented in many countries to allow for joint efforts 
between organizations.

However due to the decentralized nature of many health organizations the 
identity of a person has become a very cumbersome, costly and risky affair to handle. 
Security in a connected age, where internet is easily available, is lagging behind 
and presenting some problems. 
There is also the question of transparent use of information by the organizations 
that store it.
%
%#===========================Related Work===================================#%

\subsection{Blockchain for Identity Management in Healthcare: Use Cases}
Some companies started to see Blockchain applications in the healthcare field 
and established some key partnerships however many of those are still very early 
on development or deployment. 
One exception is Guardtime that has fully deployed 
their system for the Estonian Government in 2008 where a million patient records 
are secured by the strategy and, until today, still proves the resilience of the 
Blockchain technology thanks to other advances in the cryptography side.

Gem is a company that is collaborating with Phillips Healthcare to explore options 
in this area, and is opting to solve the interoperability problem with an additional 
layer of abstraction they call GemOS. 
Factom, another Blockchain-based service, has also announced a partnership with a 
major US medical services provider HealthNautica.\cite{BlockchainCompHealth2017,FactomPartnership2017}

%Reescrever de uma maneira mencionando medichain
%It is safe to assume that more and more companies will try to fill this void and try 
%to solve the identity problem we face.

%#==============Patient Identity Representation in Blockchain==========================#%

\section{A HLF Network for Healthcare} \label{HLFHealthcare}
\textit{After analyzing the different blockchain implementations the fabric network was 
chosen as the foundation for this system. 
First we will discuss the tools provided to configure the network, and then discuss 
the system architecture.}

\subsection{Fabric Network Configuration Tools}

HLF does not use a centralized ledger where every record would be available to every 
participant in the network. 
Instead it opts to allow multiple ledgers in a network to achieve different goals of a greater purpose. 
This allows, as discussed, the creation of channels of information between trusted parties, 
for example, a channel of secure and private information between the clinical staff of a hospital and a patient.

A HLF network has four primary tools that are used to configure the network for its purpose.

The \textit{"cryptogen"} tool generates cryptographic data consuming the file \textit{"crypto-config.yaml"}. 
HLF uses an abstraction layer for certification and authority called Membership Service Provider (MSP) 
that defines the rules by which entities are governed and authenticated and 
it must be unique for every participating entity.

The \textit{"configtxgen"} tool generates the genesis block for the orderer services and the initial transactions. 
This tool consumes the file \textit{"configtx.yaml"} that defines configuration parameters 
for channels, the genesis block and the orderer service.

The \textit{"configtxlator"} tool is also used to generate channel configurations. 
Finally the \textit{"peer"} tool is used to manage the participating peers in the HLF network.

\subsection{Identity Representation Proposal}

To manage the data of the patient they must be identified by the blockchain. 
To this purpose it is proposed that the identity of a patient is recorded on the 
ledger of the fabric network.

To aid in interoperability with other systems the FHIR standard for Health Level 7 
was used as basis for the representation of a patient.

\subsection{Aplication and Smart Contracts}

To create an interactive system an application will be built that the end-user will interact with. 
This application will interface with smart contracts through Hyperledger Fabric Software Development Kit 
built using the Hyperledger Fabric Shim for node.js.

The application will handle the user input and commands will be presented for the user to handle the application flow. 
The smart contract will handle the assets part of the system. 
A smart contract to represent and manipulate identity will be built and interface with the network 
to write records to the appropriate ledger.
\newpage
%#===========================Conclusion===================================#%

\section{Conclusion} \label{conclusion}
In this document it was described that the way identity is handled by medical institutions nowadays 
presents a problem. Blockchain was explored as a tool to solve this problem and some of its different 
implementations were analyzed. Some practical use cases of this technology also discussed. 
This research will enable solid foundations for future work.

It is safe to say that a system for improving the way a patient can interact with their health data 
can be built, using this technology, as discussed in previous sections. 
The system should allow for a transparent handling of personal data and be able to allow for secure 
management of access to this particular data.

If an advance is made in this regard it is expected that the patients trust in their Healthcare service 
is increased, and that risks and costs inherent to multiple descentralized information systems, that are 
not normalized to any standard, be reduced. 
\newpage
%#====================================Bibliography========================================#%


\begingroup
\nocite{*}
\raggedright
\bibliographystyle{alpha}
\bibliography{bibliography}
\endgroup

\end{document}